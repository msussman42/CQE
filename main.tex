%% get rid of autoindent   :setl noai nocin nosi inde=
%%
%% 1. pdflatex main
%% 2. bibtex main
%% 3. pdflatex main
%% 3. pdflatex main
%\documentclass[review]{elsarticle}
\documentclass[]{article}
%\documentclass[preprint]{elsarticle}

\usepackage[fleqn]{amsmath} 
\usepackage{amssymb}
\usepackage{graphicx}
\usepackage{tabularx} 
\usepackage{footnote}
\usepackage{titlesec}

%\journal{Journal of Computational Physics}

\newcommand{\mb}{\mathbf}
\newcommand{\tn}{\textnormal}
\newcommand{\unit}[1]{\ensuremath{\, \mathrm{#1}}}

\newcommand{\bmz}{\mbox{\boldmath $z$\unboldmath}}
\newcommand{\bmr}{\mbox{\boldmath $r$\unboldmath}}
\newcommand{\bmb}{\mbox{\boldmath $b$\unboldmath}}
\newcommand{\bms}{\mbox{\boldmath $s$\unboldmath}}
\newcommand{\bmg}{\mbox{\boldmath $g$\unboldmath}}
\newcommand{\bmq}{\mbox{\boldmath $q$\unboldmath}}
\newcommand{\bmU}{\mbox{\boldmath $U$\unboldmath}}
\newcommand{\bmu}{\mbox{\boldmath $u$\unboldmath}}
\newcommand{\bmv}{\mbox{\boldmath $v$\unboldmath}}
\newcommand{\bmw}{\mbox{\boldmath $w$\unboldmath}}
\newcommand{\bmm}{\mbox{\boldmath $m$\unboldmath}}
\newcommand{\bmi}{\mbox{\boldmath $i$\unboldmath}}
\newcommand{\bmj}{\mbox{\boldmath $j$\unboldmath}}
\newcommand{\bmk}{\mbox{\boldmath $k$\unboldmath}}
\newcommand{\bmx}{\mbox{\boldmath $x$\unboldmath}}
\newcommand{\bmX}{\mbox{\boldmath $X$\unboldmath}}
\newcommand{\bmH}{\mbox{\boldmath $H$\unboldmath}}
\newcommand{\bmy}{\mbox{\boldmath $y$\unboldmath}}
\newcommand{\bmn}{\mbox{\boldmath $n$\unboldmath}}
\newcommand{\bmf}{\mbox{\boldmath $f$\unboldmath}}
\newcommand{\bmF}{\mbox{\boldmath $F$\unboldmath}}
\newcommand{\bmS}{\mbox{\boldmath $S$\unboldmath}}
\newcommand{\bmG}{\mbox{\boldmath $G$\unboldmath}}
\newcommand{\bmV}{\mbox{\boldmath $V$\unboldmath}}
\newcommand{\bmI}{\mbox{\boldmath $I$\unboldmath}}
\newcommand{\bmD}{\mbox{\boldmath $D$\unboldmath}}
\newcommand{\bmnu}{\mbox{\boldmath $\nu$\unboldmath}}

\newcommand{\DefTen}{\mathbb{D}}
\newcommand{\EyeTen}{\mathbb{I}}

\DeclareMathOperator{\Ei}{Ei}

%%\bibliographystyle{elsarticle-num}
\bibliographystyle{acm}
%%%%%%%%%%%%%%%%%%%%%%%

\title{A review article on Computational Quantum Engineering
  \thanks{this material is based upon work supported by }}

%%!!format authors properly
%\author{
%	LastName1, FirstName1\\
%	\texttt{first1.last1@xxxxx.com}
%	\and
%	LastName2, FirstName2\\
%	\texttt{first2.last2@xxxxx.com}
%}
\author{
  Guo, Wei \\
  Mechanical Engineering, FAMU-FSU College of Engineering
  \and
  Muslimani, Ziad \\
  Mathematics, Florida State University
  \and
  Shoele, Kourosh \\
  Mechanical Engineering, FAMU-FSU College of Engineering
  \and
  Sussman, Mark \\
  Mathematics, Florida State University
}

\begin{document}
\maketitle

ABSTRACT:
The main techniques in physically storing information in a ``Quantum superposition state'' are (a) trapped ions, (b) superconducting circuits (e.g. Unimon Qubit), (c) Nitrogen-vacancy centers in diamond, and (d) Photonic Qubits.  Each of these vehicles for quantum information technology  have their advantages and disadvantages.  Important properties to consider are (i) resiliance to noise, (ii) ability to travel long distances, (iii) decoherence time, (iv) operating temperature, and (v) scalability of multi Qubit entanglement.  Scientists and Engineers answer questions regarding the aforementioned Qubit technology either experimentally or computationally.  In this review article, it is the computational methods which are reported on regarding Quantum information technology and engineering.  That being said, many numerical techniques synergistically use experimental data in order to discover quantum phenomena; i.e. many numerical methods have a data assimilation (i.e. data fusion) component.   

\linenumbers
\section{Introduction}
\cite{BLATT1967382}
``Practical points concerning the solution of the Schrödinger equation''
\cite{FEIT1982412}
``Solution of the Schrödinger equation by a spectral method''
\cite{HAN2019108929}
``Solving many-electron Schrödinger equation using deep neural networks''
\cite{https://doi.org/10.1002/andp.19273892002}
``Zur Quantentheorie der Molekeln''
\cite{BAO2002487}
``On Time-Splitting Spectral Approximations for the Schrödinger Equation in the Semiclassical Regime''
\cite{bernien2017probing}
``Probing many-body dynamics on a 51-atom quantum simulator''
\cite{zhang2017observation}
``Observation of a many-body dynamical phase transition with a 53-qubit quantum simulator''
\cite{cerezo2021variational}
``Variational quantum algorithms''
\cite{anshu2021sample}
``Sample-efficient learning of interacting quantum systems''
\cite{PhysRevLett.132.250603}
``Single-Electron Qubits Based on Quantum Ring States on Solid Neon Surface''
\cite{hermann2020deep}
``Deep-neural-network solution of the electronic Schr{\"o}dinger equation''
\cite{jiang2024high}
``High-accuracy numerical methods and convergence analysis for Schr{\"o}dinger equation with incommensurate potentials''
\cite{wang2020localization}
``Localization and delocalization of light in photonic moir{\'e} lattices''
\cite{ilin2024dissipativevariationalquantumalgorithms}
``Dissipative variational quantum algorithms for Gibbs state preparation''
\cite{somma2024shadowhamiltoniansimulation}
``Shadow Hamiltonian Simulation''
\cite{LIU2024113213}
``Dense outputs from quantum simulations''
\cite{lin2019mathematical}
``A mathematical introduction to electronic structure theory''
\cite{nielsen2010quantum}
``Quantum computation and quantum information''
\cite{ding2023simultaneous}
``Simultaneous estimation of multiple eigenvalues with short-depth quantum circuit on early fault-tolerant quantum computers''
\cite{ding2024random}
  ``Random coordinate descent: A simple alternative for optimizing parameterized quantum circuits''
\cite{LIU2024113213}
``Dense outputs from quantum simulations''
\cite{lin2016approximating}
``Approximating spectral densities of large matrices''
\cite{10628380}
``Quantum-Classical-Quantum Workflow in Quantum-HPC Middleware with GPU Acceleration''
\cite{ramkarthik2021numerical}
``Numerical recipes in quantum information theory and quantum computing: an adventure in FORTRAN 90''
\cite{biamonte2017quantum}
``Quantum machine learning''
\cite{Benedetti_2019}
``Parameterized quantum circuits as machine learning models''
\cite{PhysRevResearch.1.023025}
''Variational quantum eigensolver with fewer qubits''
\cite{saurabh2023conceptual}
``A conceptual architecture for a quantum-hpc middleware''
\cite{uvarov2020machine}
``Machine learning phase transitions with a quantum processor''
\cite{khait2023variational}
''Variational quantum eigensolvers in the era of distributed quantum computers''
\cite{huang2020predicting}
``Predicting many properties of a quantum system from very few measurements''
\cite{cong2019quantum}
``Quantum convolutional neural networks''
\cite{yeomans1988theory}
``The theory and application of axial Ising models''
\cite{parekh2021quantum}
``Quantum algorithms and simulation for parallel and distributed quantum computing''
\cite{vallero2024state}
``State of practice: evaluating GPU performance of state vector and tensor network methods''
\cite{nguyen2022tensor}
``Tensor network quantum virtual machine for simulating quantum circuits at exascale''
\cite{liu2024training}
``Training classical neural networks by quantum machine learning''
\cite{10313722}
``cuQuantum SDK: A High-Performance Library for Accelerating Quantum Science''
\cite{milburn2000ion}
``Ion trap quantum computing with warm ions''



\section{Conclusions}


\section{Conflict of interest statement }

On behalf of all authors, the corresponding author states that there is no conflict of interest.

%%\biboptions{sort&compress}
\bibliography{references}

\end{document}
%%% Local Variables:
%%% mode: latex
%%% TeX-master: t
%%% End:
